\newcommand{\DoubleImage}{true}	%Diagramms: Two on one page?

\documentclass{beamer}

%% \usepackage[ngermanb.sty]{babel}
\usepackage[utf8]{inputenc}
\usepackage{hyperref}
\usepackage{graphicx}
\usepackage{listings}
\usepackage{ifthen}

%\usepackage[demo]{graphicx}
\usepackage[compatibility=false]{caption}
\usepackage{subcaption}

\usepackage{color}
\lstdefinestyle{customc}{
  belowcaptionskip=1\baselineskip,
  breaklines=true,
%  frame=L,
  xleftmargin=\parindent,
%  language=C,
  showstringspaces=false,
  basicstyle=\footnotesize\ttfamily,
  keywordstyle=\bfseries\color{green!40!black},
  commentstyle=\itshape\color{purple!40!black},
  identifierstyle=\color{blue},
  stringstyle=\color{orange},
}

\usetheme{Berlin}		%Szeged
\usecolortheme{dolphin}	%beaver

\setbeamercovered{transparent}
\beamertemplatenavigationsymbolsempty
%\setbeamertemplate{footline}[frame number]

%% Variablen - anpassen!
\newcommand{\theauthor}{24219ff18510f65b915cbdbed33baf3d}

\newcommand{\thedate}{November 30, 2015}

\newboolean{DoublePaged} 
\setboolean{DoublePaged}{\DoubleImage} 


\title{toupper}
\subtitle{Exercise 1}
\author{\theauthor}
%\institute[LRR TUM]{LRR, I10\\ TU München}
\date{\thedate}

\begin{document}

\frame{
	\titlepage
}


\begin{frame}[fragile]
  \frametitle{toupper\_simple}
  
\begin{lstlisting}[language=c,style=customc]
static void toupper_simple(char * text) {
    char c;
    while((c=*text) != 0){
        *text++=c & 0xdf;
    }
}
\end{lstlisting}
  
\end{frame}

\begin{frame}[fragile]
  \frametitle{toupper\_128\_npw}
  
\begin{lstlisting}[language=c,style=customc]
unsigned char c[] = {0xdf};
static void toupper_128_npw(char *text){
    asm(
        ".intel_syntax noprefix             \n" 
        "   VPXOR xmm1,xmm1,xmm1            \n"
        "   VPBROADCASTB xmm2, [%1]         \n" 
        "loop_128_npw:                      \n"
        "   VPAND xmm3, xmm2, [rdi]         \n" 
        "   VPCMPISTRM xmm1, xmm3, 0x58     \n" 
        "   VMASKMOVDQU xmm3, xmm0          \n" 
        "   LEA rdi, [rdi+16]               \n" 
        "   JNZ loop_128_npw                \n"
        ".att_syntax                        \n" 
        : /* no output registers*/
        : "D"(text), "r"(c)
        : "xmm0", "xmm1", "xmm2", "xmm3");
}
\end{lstlisting}
  
\end{frame}


\begin{frame}[fragile]
  \frametitle{toupper\_128\_pw\_strm}
  
\begin{lstlisting}[language=c,style=customc]
unsigned char c[] = {0xdf};
static void toupper_128_pw_strm(char *text){
    asm(
        ".intel_syntax noprefix             \n"
        "   VPXOR xmm1,xmm1,xmm1            \n"
        "   VPBROADCASTB xmm2, [%1]         \n"
        "loop_128_pw_strm:                  \n"
        "   VPAND xmm3, xmm2, [rdi]         \n"
        "   VPCMPISTRM xmm1, xmm3, 0x58     \n"
        "   VMOVDQU [rdi], xmm3             \n"
        "   LEA rdi, [rdi+16]               \n" 
        "   JNZ loop_128_pw_strm            \n"
        ".att_syntax                        \n"
        : /* no output registers*/
        : "D"(text), "r"(c)
        : "xmm0", "xmm1", "xmm2", "xmm3");
}
\end{lstlisting}
  
\end{frame}


\begin{frame}[fragile]
  \frametitle{toupper\_128\_pw\_cmpb}
  
\begin{lstlisting}[language=c,style=customc]
unsigned char c[] = {0xdf};
static void toupper_128_pw_cmpb(char *text){
    asm(
        ".intel_syntax noprefix             \n" 
        "   VPXOR xmm1,xmm1,xmm1            \n"
        "   VPBROADCASTB xmm2, [%1]         \n"
        "loop_128_pw_cmpb:                  \n"
        "   VPAND xmm3, xmm2, [rdi]         \n"
        "   VMOVDQU [rdi], xmm3             \n"
        "   VPCMPEQB xmm0, xmm3, xmm1       \n"
        "   VPTEST xmm0, xmm0               \n"
        "   LEA rdi, [rdi+16]               \n"
        "   JZ loop_128_pw_cmpb             \n"
        ".att_syntax                        \n"
        : /* no output registers*/
        : "D"(text), "r"(c)
        : "xmm0", "xmm1", "xmm2", "xmm3");
}
\end{lstlisting}
  
\end{frame}



\begin{frame}[fragile]
  \frametitle{toupper\_256\_pw\_cmpb}
  
\begin{lstlisting}[language=c,style=customc]
unsigned char c[] = {0xdf};
static void toupper_256_pw(char *text){
    asm(
        ".intel_syntax noprefix             \n"
        "   VPXOR ymm1,ymm1,ymm1            \n"
        "   VPBROADCASTB ymm2, [%1]         \n"
        "loop_256_pw:                       \n"
        "   VPAND ymm3, ymm2, [rdi]         \n"
        "   VMOVDQU [rdi], ymm3             \n"
        "   VPCMPEQB ymm0, ymm3, ymm1       \n"
        "   VPTEST ymm0, ymm0               \n"
        "   LEA rdi, [rdi+32]               \n"
        "   JZ loop_256_pw                  \n"
        ".att_syntax                        \n"
        : /* no output registers*/
        : "D"(text), "r"(c)
        : "ymm0", "ymm1", "ymm2", "ymm3");
}
\end{lstlisting}
  
\end{frame}


\newcounter{ctra}
\setcounter{ctra}{1}
\whiledo {\value{ctra} < 5}%
{%
%\thectra {\large \ding{111}}%


\ifthenelse{\boolean{DoublePaged}}{
\begin{frame}
  \frametitle{R\the\value{ctra}}
% \includegraphics[width=0.9\textwidth]{../{r1.time}.pdf}
 
 \begin{figure}
\begin{subfigure}{.5\textwidth}
  \centering
	\includegraphics[width=\textwidth]{../{r\the\value{ctra}.time}.pdf}
  \caption{Runtime}
  \label{fig:sfig1}
\end{subfigure}%
\begin{subfigure}{.5\textwidth}
  \centering
  	\includegraphics[width=\textwidth]{../{r\the\value{ctra}.speedup}.pdf}
  \caption{Speedup}
  \label{fig:sfig2}
\end{subfigure}
%\caption{plots of R1}
\label{fig:fig}
\end{figure}
\end{frame}
}
{
\begin{frame}
  \frametitle{R\the\value{ctra} - Runtime}
\begin{figure}
\centering
	\includegraphics[width=0.9\textwidth]{../{r\the\value{ctra}.time}.pdf}
  %\caption{Runtime}
  \end{figure}
\end{frame}
\begin{frame}
  \frametitle{R\the\value{ctra} - Speedup}
\begin{figure}
\centering
	\includegraphics[width=0.9\textwidth]{../{r\the\value{ctra}.speedup}.pdf}
  %\caption{Runtime}
  \end{figure}
\end{frame}
}

\stepcounter {ctra}%
}
\end{document}
